\chapter{Leah's Chapter}


	
\section{Leah Comments Jan 24, 2024}
Please change the citation style to Author et al (year) in place of [number], so it is easier to see who you are citing without having to flip to the bibliography. Thanks for linking the bibliography to the URLs of the papers so it's possible to scan them.

When citing papers, it is best if you can also say 1-2 sentences about those papers, even based on their abstract (in your own words, of course, never copied directly). For example:

Byrne (2010) reviews theoretical cancer models and demonstrates the advantages of colaboration between modelers and experimentalists.

\subsection{Suggested research style and flavour}

Since it is unlikely that we will get data for the detailed mechanochemical mechanisms for blood vessel growth in the prostate tumors, it makes sense to (a) start simple from very minimal models that can be linked to data and (b) avoid introducing variable that we have no hope of measuing in the obtained data. My understanding is that (for now) we will have to make do with at best some bulk properties of the blood vessels, so models with a lot of detail will hardly help us.

Here is a possible minimalistic stepwise approach, where we start very simple and gradually build up more detail, starting with simple assumptions.

\textbf{Definitions:} 

$n(t)$= density of tip cells in area of interest, (number per unit area)

$\rho(t)$ = density of blood vessels (length per unit area),

$c(t)$ = concentration of drug delivered to region by blood vessels



\subsection{Step 1: Bulk model}

Ignoring spatial structure, we only track the density of vessels. Assume everything is spatially uniform, so there will be no spatial derivatives to consider. We construct an ODE model, and make an elementary assumption.

\textbf{Assumption 1:} The drug is delivered by diffusion from the capillaries into the tissue. Hence, as a rough approximation, and (for now) neglecting the detailed structure of the vessel network, the amount of drug delivered to the region per unit time is proportional to the density of the blood vessels.

\textbf{Step 1a: Elementary model:}

Assume that tips extend at some rate $v$ (units of length/time), creating additional length of capillaries as they extend. Assume capillaries may also have some loss rate $\delta$ (per unit time). Write down an ODE for the rate at which capillary density changes with time.

\begin{subequations}
	\label{eq:SimplestModel}
	\begin{equation}
		\label{eq:vesseldensODE}
		\frac{d \rho}{dt}= ??   
	\end{equation}	
	
	Assume that new tip cells are created by branching along sides of vessels (or possibly by splitting of existing tip cells) at a rate $\beta$ per unit length per unit time, and that tips disappear when they reconnect to a capillary at some rate $\kappa$ to form a loop. [Note: reconnection requires the interaction of tips with capillaries, and would thus be handled as mass action term. What are the units of $\kappa$?]
	Write down an equation for the rate of change of tip density.
	\begin{equation}
		\label{eq:tipdensODE}
		\frac{d n}{dt}= ??   
	\end{equation}	
	
	
	Complete the ODE model equations \eqref{eq:SimplestModel}. Analyse the model so far by determining the steady state densities $\rho_{ss} , n_{ss}$, and how they depend on the parameters $v,\delta,\beta, \kappa$. Determine stability of SS. Create a phase plane diagram that shows the expected dynamics. Simulate the simple ODE system assuming some values of the parameters. 
	
	We made the assumption that drug delivery is roughly proportional to the vessel density. Write down an approximate ODE for concentration of drug in the region. 
	\begin{equation}
		\label{eq:drugODE}
		\frac{d c}{dt}= ??   
	\end{equation}	
\end{subequations}Explain how this level of drug depends on the vessel branching and growth parameters.

So far, the blood vessels affect the drug but not the other way around.

\bigskip

\textbf{Step 1b: Coupling vessel dynamics to drug}

Consider how the level of drug might affect the vessel parameters (branching or growth rate or death rate, etc). This will introduce feedback from the drug to the vessel density.

Write down one or two variants of such a model and analyse them fully (including steady states, simulations, and some interpretation of what it means for overall treatment of the tissue.)

Note that the drug dynamics would be fast on the timescale of vessel growth, so there is some time-scale separation that you can take advantage of.

\subsection{(Optional) Step 2: Simple spatially distributed 1D system}
We continue with simplest model but now take spatial growth of vessels into a region. So we consider $\rho(x,t), n(x,t), c(x,t)$ as variables of interest.
We make the same assumptions as above, but now we take into account the fact that there is a flux of tips growing into a region,
\[
J=nv.
\]
Explain why this is a flux.
The equations will be modified to form PDEs. Use the 1D balance equation to create that equation for $n$. Explain whether you need to add any spatial derivatives to the equation for $\rho$. The drug diffusion in the spatial variable will also introduce spatial derivatives in the equation for $c$. Write down the modified 1D spatial model. Note that we do not assume anything like chemotaxis or other fancy mechanisms for the tip motion at this point.

\begin{subequations}
	\label{eq:SpatialtModel}
	\begin{align}
		\frac{\partial \rho}{\partial t }&= ?\\
		\frac{\partial n}{\partial t }&= ?\\
		\frac{\partial c}{\partial t }&= ?
	\end{align}
\end{subequations}

Remark: see above for timescale separation.

\textbf{Step 2a: Analysis of wave of invasion}

Consider looking for traveling wave solutions of the $\rho,n$ system on its own to ask how blood vessels spread along a 1D direction and invade a tissue. (Write down ODEs by transforming variables to $z=x-ct$ where $c$ is wave speed, then analyze existence of traveling was in the $\rho n$ phase plane. See one of my books or ask Jack Hughes for help if you are not yet familiar with this idea.)


\textbf{Step 2b: Simulations}
For simulations of the whole system: You will need to assume some boundary conditions on $n$ and on $c$, as well as some initial distribution in order to simulate this system. 



\subsection{Step 3: An agent-based (CPM) model:}
Look up the simplest work on Merks and Rens and co and find their CPM model. Ask whether a Morpheus xml file already exists for this model (can ask the Morpheus team or Merks). If not, create one.

Set up this model and adapt it to describing a simple branching vessel structure, similar to what we have above.

ADD: assume that the cells in this network ``secrete'' drug that then diffuses into the tissue and has some decay time. Find ways of plotting properties of the vessels and the drug concentration. 

Here you can get creative, and assume that the tip cell growth etc are affected by drug level, etc. (Again, time scale separation is important.) 

Your role will be to extend the Merks model to include this drug aspect.

NOTE: some of Merk's work includes the dynamics of an ECM. I would suggest to avoid extending the model with such a dynamic variable, and to assume instead, that it is a static field or vector-field that affects the rate or direction of tip cell motion.

\subsection{Step 4: Look for data}

This can be done in parallel with other steps: look for specific data on blood vessel density in normal and cancerous tissue. There may be animal studies in which the vessel density is tracked over time.

Find if there is data that we can use to help constrain any of these simple models.

For sure it's easier to find bulk vessel density than to find its spatial distribution and the chemical factors liek VEGF that are modeled in some papers.


\subsection{Step 5: More details and other variants}

You can later (after all the early steps) extend and improve the model in various ways. Some suggestions include the following:

\begin{itemize}
	\item Write down an equation for the number of loops that accumulate as tips reconnect to blood vessels (extend simple model).
	\item Find a way to associate these with ``tortuosity'' of vessel network that could affect its conductivity of drug to tissue.
	\item Consider some kind of D'Arcy's Law (porous medium) as a measure of how vessel structure can reduce net drug delivery.
	\item Vessels have various radii and sizes. You may want to consider how this affects the model as well as the implications on drug delivery. A PDE model with a distribution of vessel diameters would likely be a bit newer than the above simple branching equations.
\end{itemize}



\section{Distribution Model for Branching}

%	
\subsection{Introduction}

The purpose of the model below is to derive a continuum description of vessel density so that it will be possible to compute mean vessel radius, as well as other aggregate properties that can, in turn, help to approximate the flow of blood through tissue.

Some questions that this model could help answer are: how do tip branching and fusing combine to determine the properties of the vessels in some average way. How is mean radius dependent on assumptions about branching and fusing? If drugs or treatments affect the mortality of vessels, or the branching, how would the aggregate properties change?


\subsection{Derivation of the model}

Let $n(r,t), \rho(r,t)$ be the density of tips of blood vessels and of vessels that have radius $r$ at time $t$. That is,
\[
\int_a^b n(r,t) \, dr = \text{number of tips of vessels whose radius is in } a \le r \le b
\]
and similarly for $\rho$.
Note that the total number of tips and the total length of vessels in the ``unit volume'' would then be
\[
N=\int_0^\infty n(r,t) \, dr, \quad  P=\int_0^\infty \rho(r,t) \, dr
\]
and the mean radius of the vessels in the unit volume would be
\[
r_\text{mean}=\int_0^\infty r \, \rho(r,t) \, dr
\]
We can similarly get higher moments of the distribution. In practice, there is some maximal radius, which is the upper limit of the integral. 

%\begin{figure}
%	\centering
%	\includegraphics[width=0.5\linewidth]{Figures/SimpleTubesVesselModel.png}
%	\caption{The simple model for radial distribution of vessels can provide the most basic first idea of a block of tissue as a porous medium through which blood can flow. The model can inform us about the distribution of ``pores'' (radial vessels of various sizes). Left: the model keeps track of vessels of various radii in a unit volume of tissue. Right: view from another side of the block is basically that of a lot of ``holes'', like in porous medium. }
%	\label{fig:SimpleTubesVesselModel}
%\end{figure}

We consider the following processes ad rates
\begin{align*}
	\beta(r)&= \text{branching rate,}\\
	K(r,R) &= \text{probability that tip of a vessel of radius } R \text{ produces a daughter of radius } r, \\
	\phi &= \text {average number of daughters produced per branching event},\\
	A(r,R)&= \text{probability of fusing of tip of radius }r \text{ with vessel of radius }R,\\
	\psi(r)&= \text{anastomosis rate},\\
	v(r)&= \text{rate of tip extension},\\
	\gamma(r) &= \text{rate of vessel mortality}.    
\end{align*}

Then a model for $n,\rho$ would be
\begin{align}
	\frac{\partial n}{\partial t} &= -\beta(r) n + \phi \beta(r) \int K(r,R) n(R,t) \, dR - \psi n(r,t) \int A(R,r)\rho(R,t)\, dR\\
	\frac{\partial \rho}{\partial t} &= v(r) n(r,t) - \gamma(r) \rho(r,t)  .
\end{align}

In the first equation, the terms are (a) loss of a tip of radius $r$ when it splits into daughters of other radii, (b) creation of new tip of radius $r$ from any other parents of radius $R$. The integral sums over all such possible parents, with a kernel $K(R,r)$ that weights the contribution from $R$-radius parents to $r$-radius daughters. ($K$ is assumed to be normalized; more will be said about this kernel later.) A branching event produces 1 or more daughters, in general, and $\phi$ is that average number of daughters. ($\phi \approx 2$ typically.) (c) The third term is loss of tip of radius $r$ due to fusing with any other branches of radius $R$. The kernel $A$ (``anastomosis''), also normalized, gives the probability of this event, and $\psi>0$ is some rate.


In the second equation, density of $r$-radius vessels increases when their tips grow, and we assume a low rate of decay to allow for some steady state.
The vessels grow in length at their tips (which deposits branch length behind them). But it is assumed that they do not widen radially. (If this does happen, there would be a term of the form $-\partial (n \omega)/\partial r$ where $\omega$ is the rate of radial growth on the RHS of the first equation to take that into account.)

The above model is nonlocal in that the dynamics of vessels at one radius (say $r$) depend on vessels at far larger (or smaller) radii. The kernels in such nonlocal equations will be important in the predicted behaviour of the system. However, the model is currently not spatial. It could be extended in that was, similar to \cite{edelstein1989models}, but we first consider this nonspatial version. For models using nonlocal equations (PDEs with integral terms) we refer the reader to papers on angular distributions such as \cite{edelstein1990contact,edelstein1990models,mogilner1995selecting,mogilner1995modelling,mogilner1999non,edelstein2005mathematical}

In principle, many of the coefficients may be dependent on vessel size. That dependence will greatly influence the outcome of the model. However, most of the analysis will be restricted to simple special cases where those rates are constant.

\subsection{Possible kernels and their interpretation}

\subsection{Branching kernel}

We can think of several simplifying cases for which the model can be analysed, including:

\begin{itemize}
	\item[k1:] $K$ is constant. (This is not very realistic, as it implies that a vessel can branch into daughters of any radii, including larger radii, which is implausible.
	\item[k2:] $K(r,R)$ is a delta-function, 
	$K=\delta(r-\alpha R)$. This means that parents of radius $R$ produce only daughters of radius $r=\alpha R$, where it makes sense that $0\le \alpha \le 1$.
	\item[k3:] $K(r,R)$ is some Gaussian with mean at $r=\alpha R$. This last case can be simulated numerically.
\end{itemize}
The cases k1, k2 lead to some possible analysis.

\subsection{Branching rate}
It is possible that the vessel size affects the branching rate: possibly $\beta(r)$ is an increasing or decreasing function of radius. We will consider $\beta \ge 0 =$ constant in some cases that can be analysed.

\subsection{Anastomosis kernel}

It is not clear whether fusion of a tip with a vessel depends on the radii of these two. As in the case of the branching kernel, there are several possible assumptions, including

\begin{itemize}
	\item[a1:] $A$ is constant. (This is not very realistic, as it implies that a vessel of radius $r$ has an equal probability of fusing with a vessel of any other size. 
	\item[a2:] $A(r,R)$ is a delta-function, 
	$K=\delta(r-R)$. This means that an $r$-radius tip can only fuse to an $R$-radius vessel if $R=r$ radius.
	\item[a3:] $A(r,R)$ is some Gaussian with mean at $r=R$. 
\end{itemize}


%%%%%%%%%%%%%%%%%%
\subsection{Analysis of special cases}

Below, we consider a number of simple situations where, under suitable assumptions, the model can be analysed either qualitatively or quantitatively. Even though many simplifications are somewhat unrealistic, we can use these results to get initial insights into how the branching system behaves.

\subsection{Case 1: No anastomosis}
Consider the case that (1) there is insignificant tip fusions, so that $A\approx 0$ and (2) Branching of $r$-sized tip produces daughters of any size smaller than $r$, with equal probability, that is $\beta=\beta(R)$ depends only on the parent radius, $R$, with $K(R,r)=H(R-r)$ a step function that ``turns on'' when $R=r$. (3) There is also some (possibly size-dependent) rate of tip stalling (or death), $\kappa(r)\ge 0$.
Then the model has the form
\begin{equation}
	\label{eq:Case1:noAnast}
	\frac{\partial n}{\partial t} = -\beta(r) n + \phi  \int_r^\infty \beta(R) n(R,t) \, dR - \kappa(r) n(r,t)     
\end{equation}
The first term is as before, while in the second term, we took the branching kernel to be simpler. The last term is 1st order decay of tips.

This model can be analyzed by a method used in section 4 of \cite{edelstein2001model}.Look for a steady state distribution where $\frac{\partial n}{\partial t}=0$, and define
\[
z(r)= \int_r^\infty \beta(R) n(R,t) \, dR
\]
Then $\phi z$ is a cumulative rate of branching that results in $r$ sized vessels.

By the Fundamental Theorem of Calculus,
\[
\frac{dz}{dr}= -\beta(r) n(r)
\]
(where we have used an assumption that $n(r)\to 0$ as $r\to \infty$). We can now rewrite the steady state version of Eqn.~\eqref{eq:Case1:noAnast} in the form
\[
0=\frac{dz}{dr} + \phi z +\frac{\kappa(r)}{\beta(r)} \frac{dz}{dr}
\]
where we have replaced each term by the equivalent term in $z$. But this is just a simple first order ODE for $z(r)$,
\[
\frac{dz}{dr}=- \frac{\phi \beta(r)}{\beta(r)+\kappa(r)} \, z.
\]
This can be easily integrated, by the usual technique of separation, i.e.,
\[
\frac{dz}{z}=- \frac{\phi \beta(r)}{\beta(r)+\kappa(r)} \, dr.
\]
It would be interesting to consider this solution for various specific assumptions about the functions $\beta(r),\kappa(r)$. We can then get an analytic formula for $n(r)$, and also for the vessel distribution $\rho(r)$.
For example, if $\beta,\kappa$ are constants, we immediately see an exponential distribution, as expected.



\subsection{Case 2: Delta function kernels}

One special case that can be analyzed is that of delta function kernels $K(r,R)=\delta(r-\alpha R), A(r, R) = \delta(r-R)$. 
We also assume that all coefficients are constant parameters.
In this case, the model equations reduce to
\begin{align}
	\frac{\partial n}{\partial t} &= -\beta n + \phi \beta n(r/\alpha,t)  - \psi n(r,t) \rho(r,t)\\
	\frac{\partial \rho}{\partial t} &= v n(r,t) - \gamma \rho(r,t)  .
\end{align}
That is, branching or parents with radius $R$ produces daughters at radius $r=\alpha R$, and fusion is only with vessels of the same radius.

The steady state of this system satisfies
\begin{align}
	0 &= -\beta n(r) + \phi \beta n(r/\alpha)  - \psi n(r) \rho(r)\\
	0 &= nv(r) - \gamma \rho(r)  .
\end{align}
We can eliminate $\rho$ using the second equation, obtaining
\begin{equation}
	\label{eq:discreteEq1}
	0= -\beta n(r) + \phi \beta n(r/\alpha)  - \frac{v\psi}{\gamma} n^2(r)
\end{equation}
This equation links daughter tips of radius $r$ to parent tips whose (larger) radius is $r/\alpha$. Let us define the ith generation as the parents and the i+1 generation as the daughters. Assign the discrete variables
\begin{align}
	n_{i+1}&= n(r)\\
	n_i&=n(r/\alpha).
\end{align}
Then the above equation can be written as
\begin{equation}
	\label{eq:discreteEq2}
	0= -\beta n_{i+1} + \phi \beta n_i - \frac{v\psi}{\gamma} n^2_{i+1}
\end{equation}
This is a nonlinear discrete dynamical ``map'', let us rewrite it as
\[
n_i=f(n_{i+1}), \quad \text{where} \quad f(x)=\frac{1}{\phi}\left(x+\frac{\psi}{\beta}x^2\right).
\]
The ``generator'' of this iterated map is a quadratic function.  While we cannot solve it in closed form, we can use the usual ``cobwebbing'' method to solve it graphically, as shown in \cite{edelstein2005mathematical}. We can also iterate it on a computer. 

We can also determine conditions on the parameters for which there is a progression $n_i \to 0$ as the generation number increases, or where $n_i \to N_0$, where $N_0>0$ is some constant tip density.

Note: this map is related to the famous Robert May's discrete logistic map, but our iteration goes in reverse, from i+1 to i. That does not matter in the ability to understand the qualitative behaviour of the system.


