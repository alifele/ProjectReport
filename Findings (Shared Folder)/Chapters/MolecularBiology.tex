\chapter{Molecular Biology}
Here in this chapter, I will be covering the basics of the relevant molecular biology concepts. This chapter will serve as a reference for the biological claims throughout the document, as well as the foundation for the review chapters of my thesis.

\section{Molecular Mechanism of Angiogenesis}
There are two general balancing forces acting on the angiogenesis
\begin{itemize}
	\item Inhibitors:
	\begin{itemize}
		\item endostatin
		\item angiostatin
		\item thrombospondin
	\end{itemize} 
	\item Angiogens
	\begin{itemize}
		\item VEGF: Vascular Endothelial Growth Factors.
		\item bFGF: Basic Fibroblast Growth Factor.
		\item PDGF: Platelet Driven Growth Factor.
	\end{itemize}
\end{itemize}


\subsubsection*{Controlling Capillary Joining Process}
In the following text from \cite{Alberts2002}, there is some vague hints about the mechanisms that are controlling capillary joining to each other.


\begin{quote}
	Observations such as these reveal that endothelial cells that are to form a new capillary grow out from the side of an existing capillary or small venule by extending long pseudopodia, pioneering the formation of a capillary sprout that hollows out to form a tube (Figure 22-25). This process continues until the sprout encounters another capillary, with which it connects, allowing blood to circulate. Endothelial cells on the arterial and venous sides of the developing network of vessels differ in their surface properties, in the embryo at least: the plasma membranes of the arterial cells contain the transmembrane protein ephrin-B2 (see Chapter 15), while the membranes of the venous cells contain the corresponding receptor protein, Eph-B4, which is a receptor tyrosine kinase (discussed in Chapter 15). These molecules mediate a signal delivered at sites of cell-cell contact, and they are essential for the development of a properly organized network of vessels. One suggestion is that they somehow define the rules for joining one piece of growing capillary tube to another.

\end{quote}

\subsubsection*{Formation of tube structures by endothelial cells}
It was one of my main concerns that what is the process in which a single lining of endothelial cells following a tip cell forms a hallow tube (i.e. vessel). The following text from \cite{Alberts2002} explains this clearly. This process has also been described in \cite{angiogenesisYoutube}.
\begin{quote}
	Experiments in culture show that endothelial cells in a medium containing suitable growth factors will spontaneously form capillary tubes, even if they are isolated from all other types of cells (Figure 22-26). The capillary tubes that develop do not contain blood, and nothing travels through them, indicating that blood flow and pressure are not required for the initiation of a new capillary network.
	
	Endothelial cells in culture spontaneously develop internal vacuoles that appear to join up from cell to cell, giving rise to a network of capillary tubes. These photographs show successive stages in the process. 
\end{quote}






\section{Biological Assays to Study Angiogenesis}

\subsection{Corneal Micropocket Assay}
This is one of the simple and reproducible assays to study angiogenesis in a eye. The process involves introducing growth factors in the eye ball of mouse, and then letting the vascular network to form. This is a video from JOVE explaining the details of the protocol \citep{conealMicroPocketAssayJOVE} 