\chapter{Statistical Physics and Biology}


\section{Introduction}

The future of medicine, Time magazine, Jan 1999: ``Ring farewell to the century of physics, the one on which we split the atom and turned silicon into computing power. It's time to ring in the century of biotechnology \cite{Sung2018}.'' Despite thee tremendous importance of life science and biotechnology nowadays as the above statements proclaim, at this stage their knowledge appears to be largely phenomenological, and thus undeniably calls for fundamental and quantitative understandings of the complex phenomena. It will be timely to ring in the century of a new physical science to meet this challenge.

The concept of self-organization is a central theme in biology and there are many great review paper on the subject (like \cite{Pochan2021,McManus2016,Whitelam2015}). In short \textbf{biological components self-assemble themselves to function. To perform the biological self-organization, then oftern cross over the energy barriers that seem to be insurmountable in the view point of simple physics. To this end, there are two physical characteristics that feature in the mesoscopic biological systems introduced above. The first one is their aqueous environment and thermal fluctuations therein}. The water has many outstanding properties among all liquids. It has a very high heat capacity, meaning that it can act as a heat reservoir with a negligible temperature change. Also, it has a very high (compared to other liquids) di-electric constant. Because of this, water can reduce electrostatic energy of the interactions to the level of thermal energy. This unique property of water originates from the weak hydrogen bonds between molecules. Although, the hydrogen bounds can be broken with the thermal energy of the environment, it causes long range correlation between water molecules. As a result, the liquid water manifests a quasi-critical state where it responds collectively and sensitively to external stimuli \cite{Sung2018,Wiggins1990,Bagchi2013}. The biological systems in mesoscale characterized by the \textit{soft inter connectivity and weak interactions} may appropriately be called the \textbf{bio-soft condensed matter} \cite{Sung2018}. 







\section{Keywords}
Quasi-critical state, collective behaviour, self-assembly, bio-soft condensed matter, Coarse graining, non-equilibrium thermodynamics, 