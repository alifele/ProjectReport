
%\documentclass[12pt,a4paper]{book}
%\usepackage{commands}
%
%\begin{document}
%
%	
%\end{document}
\documentclass{labdiary}
\usepackage{commands}
\usepackage[numbers]{natbib} 

\title{My Findings}
\author{Ali Fele Paranj}
\date{\today}

\begin{document}
	
	\maketitle
	
	\tableofcontents
	
	\chapter{Statistical Physics and Biology}


\section{Introduction}

The future of medicine, Time magazine, Jan 1999: ``Ring farewell to the century of physics, the one on which we split the atom and turned silicon into computing power. It's time to ring in the century of biotechnology \cite{Sung2018}.'' Despite thee tremendous importance of life science and biotechnology nowadays as the above statements proclaim, at this stage their knowledge appears to be largely phenomenological, and thus undeniably calls for fundamental and quantitative understandings of the complex phenomena. It will be timely to ring in the century of a new physical science to meet this challenge.

The concept of self-organization is a central theme in biology and there are many great review paper on the subject (like \cite{Pochan2021,McManus2016,Whitelam2015}). In short \textbf{biological components self-assemble themselves to function. To perform the biological self-organization, then oftern cross over the energy barriers that seem to be insurmountable in the view point of simple physics. To this end, there are two physical characteristics that feature in the mesoscopic biological systems introduced above. The first one is their aqueous environment and thermal fluctuations therein}. The water has many outstanding properties among all liquids. It has a very high heat capacity, meaning that it can act as a heat reservoir with a negligible temperature change. Also, it has a very high (compared to other liquids) di-electric constant. Because of this, water can reduce electrostatic energy of the interactions to the level of thermal energy. This unique property of water originates from the weak hydrogen bonds between molecules. Although, the hydrogen bounds can be broken with the thermal energy of the environment, it causes long range correlation between water molecules. As a result, the liquid water manifests a quasi-critical state where it responds collectively and sensitively to external stimuli \cite{Sung2018,Wiggins1990,Bagchi2013}. The biological systems in mesoscale characterized by the \textit{soft inter connectivity and weak interactions} may appropriately be called the \textbf{bio-soft condensed matter} \cite{Sung2018}. 







\section{Keywords}
Quasi-critical state, collective behaviour, self-assembly, bio-soft condensed matter, Coarse graining, non-equilibrium thermodynamics, 
	\chapter{Papers Reviewed}


% Redefine section command for this chapter
\let\oldsection\section
\renewcommand{\section}{\newreviewsection}

% Your custom sections here

In this section I will keep the notes of the papers I have reviewed, or reproduced their results. 



\section{Topological data analysis distinguishes parameter regimes in the Anderson-Chaplain model of angiogenesis}{Nardini, Byrne}{2021-PLOS CB}


\subsection{Introduction}
This paper studies the Anderson Chaplain \cite{Anderson1998} model of angiogenesis and partitions the parameter spaces based on the morphology of the vascular structure generated by the model. In other words, let $P = R^d$ be the parameter space of the model, $M$ the space of all possible morphology for the vascular networks. Also, define the equivalence relation $\sim$ defined on the parameter space $P$ to be
\[ \text{for } p_1, p_2 \in P \wh p_1 \sim p_2 \qquad iff \qquad \mathcal{A}(p_1) \equiv \mathcal{A}(p_2), \]
where $\mathcal{A}: P \to M$ a mapping from the parameter space to the morphology space. The $\equiv$ is yet another equivalence relation defined on the morphology space $M$ where for $m_1, m_2 \in M$ we write $m_1 \equiv m_2$ if and only if $m_1$ and $m_2$ has the same topological characterization. These topological characterizations are computed using the topological data analysis techniques.


\subsection{Method}
Chaplain-Anderson model of angiogenesis used in this paper keeps track of the spatio-temporal evolution of three variables: endothelial tip cells, tumor angiogenesis factor, and fibronectin.

Topological data analysis: Two filtration methods were used: sweeping plane method, and flooding filtration. The filtration is performed on the binary images generated with the Chaplain-Anderson model.


\subsection{Useful facts}
\begin{itemize}
	\item The growth factors the cancer cells release when under low nutritient and oxygen: vascular endothelial growth factor (VEGF), platelet derived growth factor (PDGF), and basic fibroblast growth factor (bFGF).
\end{itemize}

\subsection{Points that are not clear yet}
\begin{enumerate}[(a)]
	\item In the introduction, the authors claim that ``The morphology of a vascular network can reveal the presence of an underlying disease, or predict the response of a patient to treatment'', without any citation of explanation. I think this needs more discussion.
\end{enumerate}

\subsection{Useful papers cited}
\begin{itemize}
	\item Papers related to biology of the tumor induced angiogenesis \cite{Gupta2003,Folkman1971}.
	\item More modern descriptions of the angiogenesis \cite{Lugano2020,Saman2020}
	\item The role of the mechanical stress on the angiogenesis \cite{Li2005,Li2002,Vavourakis2017}
	\item Some old and classic models for the angiogenesis \cite{Anderson1998,Balding1985,Byrne1995,Stokes1991}.
	\item More detailed theoretical models for angiogenesis \cite{Byrne2010,Hadjicharalambous2021,Metzcar2019,Scianna2013}.
	\item Alternative models of angiogenesis \cite{Vilanova2017,Stepanova2021,Perfahl2017,Grogan2017,Vavourakis2017,Cai2017,Sefidgar2015}
	\item Statistical and single scale methods to quantify the vascular networks \cite{Perfahl2017,Folarin2010,Kannan2018,Konerding1999,Konerding2001}
	\item Biological angiogenesis experiments \cite{Bauer2007}
%	\item Topological data analysis \cite{Carlsson2009,Zomorodian2005}
%	\item Applied Algebraic Topology \cite{2020,2023}
%	\item Application of topological data analysis \cite{Nicolau2011,Nielson2015,Qaiser2016,Xia2014}
%	\item Topological data analysis and agent based models \cite{Topaz2015,McGuirl2020}
\end{itemize}




\subsection{Results}




	
	
	

% Reset section command back to original after the chapter
\let\section\oldsection

	
	
	
\section{Leah Comments Jan 24, 2024}
Please change the citation style to Author et al (year) in place of [number], so it is easier to see who you are citing without having to flip to the bibliography. Thanks for linking the bibliography to the URLs of the papers so it's possible to scan them.

When citing papers, it is best if you can also say 1-2 sentences about those papers, even based on their abstract (in your own words, of course, never copied directly). For example:

Byrne (2010) reviews theoretical cancer models and demonstrates the advantages of colaboration between modelers and experimentalists.

\subsection{Suggested research style and flavour}

Since it is unlikely that we will get data for the detailed mechanochemical mechanisms for blood vessel growth in the prostate tumors, it makes sense to (a) start simple from very minimal models that can be linked to data and (b) avoid introducing variable that we have no hope of measuing in the obtained data. My understanding is that (for now) we will have to make do with at best some bulk properties of the blood vessels, so models with a lot of detail will hardly help us.

Here is a possible minimalistic stepwise approach, where we start very simple and gradually build up more detail, starting with simple assumptions.

\textbf{Definitions:} 

$n(t)$= density of tip cells in area of interest, (number per unit area)

$\rho(t)$ = density of blood vessels (length per unit area),

$c(t)$ = concentration of drug delivered to region by blood vessels



\subsection{Step 1: Bulk model}

Ignoring spatial structure, we only track the density of vessels. Assume everything is spatially uniform, so there will be no spatial derivatives to consider. We construct an ODE model, and make an elementary assumption.

\textbf{Assumption 1:} The drug is delivered by diffusion from the capillaries into the tissue. Hence, as a rough approximation, and (for now) neglecting the detailed structure of the vessel network, the amount of drug delivered to the region per unit time is proportional to the density of the blood vessels.

\textbf{Step 1a: Elementary model:}

Assume that tips extend at some rate $v$ (units of length/time), creating additional length of capillaries as they extend. Assume capillaries may also have some loss rate $\delta$ (per unit time). Write down an ODE for the rate at which capillary density changes with time.

\begin{subequations}
\label{eq:SimplestModel}
\begin{equation}
\label{eq:vesseldensODE}
 \frac{d \rho}{dt}= ??   
\end{equation}	
	
Assume that new tip cells are created by branching along sides of vessels (or possibly by splitting of existing tip cells) at a rate $\beta$ per unit length per unit time, and that tips disappear when they reconnect to a capillary at some rate $\kappa$ to form a loop. [Note: reconnection requires the interaction of tips with capillaries, and would thus be handled as mass action term. What are the units of $\kappa$?]
Write down an equation for the rate of change of tip density.
\begin{equation}
\label{eq:tipdensODE}
 \frac{d n}{dt}= ??   
\end{equation}	


Complete the ODE model equations \eqref{eq:SimplestModel}. Analyse the model so far by determining the steady state densities $\rho_{ss} , n_{ss}$, and how they depend on the parameters $v,\delta,\beta, \kappa$. Determine stability of SS. Create a phase plane diagram that shows the expected dynamics. Simulate the simple ODE system assuming some values of the parameters. 

We made the assumption that drug delivery is roughly proportional to the vessel density. Write down an approximate ODE for concentration of drug in the region. 
\begin{equation}
\label{eq:drugODE}
 \frac{d c}{dt}= ??   
\end{equation}	
\end{subequations}Explain how this level of drug depends on the vessel branching and growth parameters.

So far, the blood vessels affect the drug but not the other way around.

\bigskip

\textbf{Step 1b: Coupling vessel dynamics to drug}

Consider how the level of drug might affect the vessel parameters (branching or growth rate or death rate, etc). This will introduce feedback from the drug to the vessel density.

Write down one or two variants of such a model and analyse them fully (including steady states, simulations, and some interpretation of what it means for overall treatment of the tissue.)

Note that the drug dynamics would be fast on the timescale of vessel growth, so there is some time-scale separation that you can take advantage of.

\subsection{(Optional) Step 2: Simple spatially distributed 1D system}
We continue with simplest model but now take spatial growth of vessels into a region. So we consider $\rho(x,t), n(x,t), c(x,t)$ as variables of interest.
We make the same assumptions as above, but now we take into account the fact that there is a flux of tips growing into a region,
\[
J=nv.
\]
Explain why this is a flux.
The equations will be modified to form PDEs. Use the 1D balance equation to create that equation for $n$. Explain whether you need to add any spatial derivatives to the equation for $\rho$. The drug diffusion in the spatial variable will also introduce spatial derivatives in the equation for $c$. Write down the modified 1D spatial model. Note that we do not assume anything like chemotaxis or other fancy mechanisms for the tip motion at this point.

\begin{subequations}
\label{eq:SpatialtModel}
\begin{align}
\frac{\partial \rho}{\partial t }&= ?\\
\frac{\partial n}{\partial t }&= ?\\
\frac{\partial c}{\partial t }&= ?
\end{align}
\end{subequations}

Remark: see above for timescale separation.

\textbf{Step 2a: Analysis of wave of invasion}

Consider looking for traveling wave solutions of the $\rho,n$ system on its own to ask how blood vessels spread along a 1D direction and invade a tissue. (Write down ODEs by transforming variables to $z=x-ct$ where $c$ is wave speed, then analyze existence of traveling was in the $\rho n$ phase plane. See one of my books or ask Jack Hughes for help if you are not yet familiar with this idea.)


\textbf{Step 2b: Simulations}
For simulations of the whole system: You will need to assume some boundary conditions on $n$ and on $c$, as well as some initial distribution in order to simulate this system. 



\subsection{Step 3: An agent-based (CPM) model:}
Look up the simplest work on Merks and Rens and co and find their CPM model. Ask whether a Morpheus xml file already exists for this model (can ask the Morpheus team or Merks). If not, create one.

Set up this model and adapt it to describing a simple branching vessel structure, similar to what we have above.

ADD: assume that the cells in this network ``secrete'' drug that then diffuses into the tissue and has some decay time. Find ways of plotting properties of the vessels and the drug concentration. 

Here you can get creative, and assume that the tip cell growth etc are affected by drug level, etc. (Again, time scale separation is important.) 

Your role will be to extend the Merks model to include this drug aspect.

NOTE: some of Merk's work includes the dynamics of an ECM. I would suggest to avoid extending the model with such a dynamic variable, and to assume instead, that it is a static field or vector-field that affects the rate or direction of tip cell motion.

\subsection{Step 4: Look for data}

This can be done in parallel with other steps: look for specific data on blood vessel density in normal and cancerous tissue. There may be animal studies in which the vessel density is tracked over time.

Find if there is data that we can use to help constrain any of these simple models.

For sure it's easier to find bulk vessel density than to find its spatial distribution and the chemical factors liek VEGF that are modeled in some papers.


\subsection{Step 5: More details and other variants}

You can later (after all the early steps) extend and improve the model in various ways. Some suggestions include the following:

\begin{itemize}
    \item Write down an equation for the number of loops that accumulate as tips reconnect to blood vessels (extend simple model).
    \item Find a way to associate these with ``tortuosity'' of vessel network that could affect its conductivity of drug to tissue.
    \item Consider some kind of D'Arcy's Law (porous medium) as a measure of how vessel structure can reduce net drug delivery.
    \item Vessels have various radii and sizes. You may want to consider how this affects the model as well as the implications on drug delivery. A PDE model with a distribution of vessel diameters would likely be a bit newer than the above simple branching equations.
\end{itemize}



	%\backmatter
	% Appendices, bibliography, etc.
	
	% Bibliography
	\bibliographystyle{unsrtnat} % The bibliography style
	\bibliography{References/references} % The filename of the .bib file (without the extension)

	

	
	
\end{document}
