\chapter{Meeting log}
\section{Meetings with Leah}
\subsection{29 Jan Meeting}

\begin{itemize}[itemsep=0pt, parsep=0pt]
	\item Fixing some errors in the eigenvalues for the main differential equations. 
	\item Add the nullclines plot.
	\item Add the possible interactions between vascular networks and the drug
\end{itemize}


\subsection{5 Feb Meeting}
\begin{itemize}[itemsep=0pt, parsep=0pt]
	\item Thinking again about including the decay of radiopharmaceuticals: This has both pros and cons. The pros is that 
	\begin{itemize}[itemsep=0pt, parsep=0pt]
		\item more realistic model,
		\item it is always good to have same sort of decay in the model to ensure the stability.
	\end{itemize}
	However, the downside is that radiation interacts more wildly with the the cells present in the mode. It can inhibit them (by killing them) which is in a non mass action or Hill function style. The killing mechanism follows the linear quadratic rule. Also, the radiation can have activation functionality on the same cells by simply causing crazy genetic mutations. \textbf{So decision on including the radiation term in the basic model should be done with extra care.}
	\item Doing qualitative analysis (not quantitative) with the nullclines and the change in parameters due to the drug interaction. This way we can capture the possible interactions with out considering the actual functional form of the interaction.
	\item After doing the qualitative analysis, I need to do some literature review to see what are the possible drug-vessel interactions (both radiation and chemical)
	\item Consider adding the tumor compartment. The tumor compartment can interact with the vascular network by
	\begin{itemize}[itemsep=0pt, parsep=0pt]
		\item increasing the mobility of the tip cells: both by increasing chemotaxis agents, and also by loosens the extracellular matrix,
		\item Any other interaction that needs to be determined carefully.
	\end{itemize}
	\item Adding the condition under which the stability of $p^0_2$ is focus or node.
\end{itemize}


\subsection{12 Feb Meeting}
\begin{itemize}
	\item Add the axis labels for the qualitative analysis, and edit the title.
	\item Consider the non-dimensionalization process to see if you can control the non-linear term.
	\item Add the section for drug-tissue interaction. Consider the possible simplifications of the model.
\end{itemize}

\subsection{8 March Meeting}
\begin{itemize}
	\item To develop the PDEs for the model: Note the following facts: The blood vessels are not moving, but they are deposited by the moving endothelial cells.
	\item There should be no diffusion or advection terms in the PDE for $ \rho $.
	\item Although the tip cells are diffusing as well, but we can consider their dynamics to be advection dominated diffusion. Thus no diffusion term at the moment.
	\item The advection speed of the tip cells is the same as the speed that they move in the bulk model.
	\item Try the traveling wave solution to the derived PDE.
	\item Be careful about the terms that can be simplified by the time scale separation argument. 
\end{itemize}

\subsection{12 March Meeting}
\begin{itemize}
	\item Things to discuss: going through the qualitative study once again, discussing why we can not have $ \kappa $ directly appearing in the non-dimensionalized version, discussing the effect of $ \kappa $ on the dynamics.
	
	\item 
	\[ \frac{\partial n}{\partial t} = D \frac{\partial^2 n}{\partial x^2} - \frac{\partial(nv)}{\partial x} + (\lambda_s - \delta_t)n - \kappa_2 n^2. \]
	This can be thought of as a model that $ \lambda_b \to 0 $.
\end{itemize}

\subsection{20 March Meeting}
\begin{itemize}
	\item Jupiter: More analysis on the vertical asymptotic.
	\item Laurent: Drawing a diagram summerizing all of the different interactions happening
	\begin{itemize}
		\item Effect of drug on vasculature
		\item Effect of vasculature on tumor response
		\item Effect of vasculature on drug delivery
		\item Effect of tumor on all of these
		\item what is the topology of the reaction graph?
	\end{itemize}
	\item Jupiter: adaptive systems
	\item Leah: (1) I agree that vessel density is a good metric, and I suppose that often this is more easy to measure in scans of images than any other network property. I seriously doubt that you can estimate the number of loops or tips in 2D images for all but the very low density of vessels.
	\begin{enumerate}[(i)]
	
	\item Eric asked: How do veins and arteries connect up. Good to look into this.
	
	
	
	\item He asked what causes more/less resistance to radiotherapy. Eric asked if you were thinking about devising some fractionation protocol that would result in tumor reduction overall.
	
	There was also a question about the effect of hypoxia - pro or anti tumor, pro or anti vessel growth etc.
	
	\item Tim asked if you were considering vessels to be tubes. I think it's OK to assume that the tubes emerge once the branches grow.
	
	\item Eric was asking what the model is describing, e.g. what timescale, etc. Do the vessels degrade over such timescale? Or do you want to consider a model where the growth rate v goes to 0 at steady state? [LEK: we are currently exploring a basic simple model with all parameters assumed to be constants.. but the eventual goal is to include the effect of tumor, and drug.. etc, and feedback between treatment and vessels.]
	
	\item You did not yet mention the drug c, and how you might include it.
	
	\item Talk to Tim about his microtubule branching ABM.
	
	\item Do you really want to nondimensionalize the model? Jupiter as confused why n --> 1 always. It's due to the scaling. If you keep the original parameters, you can see how they affect the n and rho steady states.
	
	\item Jimmy said so far you focused on the angiogenesis, what about the tumor. It would be good to mention that this will be considered later.
	
	\item Laurent: will drug feed back on vasculature? will tumor affect vasculature?
	
	He and Jack also asked if tips diffuse.
	
	[We do not want to redo the model by the Helen Byre group, but we could include later the chemotaxis of tips towards the tumor, as well as a bit of tip diffusion.]
	
	\item Talk to Jupiter about adaptive systems.
	
	
	\item Discussion about TA.
	
	
	\item General comments from me:
	\begin{itemize}
	
	\item On every slide you show, you need to give credit to where you got images, tables, data, graphs etc that you did not create yourself, even if available on the web. It can be in small font, but it is essential in all talks.
	
	\item I know you are still learning about TDA. But I am doubtful whether it will be helpful to us since (1) we have few images, if any to analyse (2) the images are all in 2D, so vessels overlap or disappear into the 3rd dimension (what you called blunt ends) making it tricky or impossible to detect true loops or other features. Density seems reasonable.
	\end{itemize}
\end{enumerate}
\end{itemize}

\section{Meetings with Arman}