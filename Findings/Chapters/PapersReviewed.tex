\chapter{Papers Reviewed}


% Redefine section command for this chapter
\let\oldsection\section
\renewcommand{\section}{\newreviewsection}

% Your custom sections here

In this section I will keep the notes of the papers I have reviewed, or reproduced their results. 



\section{Topological data analysis distinguishes parameter regimes in the Anderson-Chaplain model of angiogenesis}{Nardini, Byrne}{2021-PLOS CB}


\subsection{Introduction}
This paper studies the Anderson Chaplain \cite{Anderson1998} model of angiogenesis and partitions the parameter spaces based on the morphology of the vascular structure generated by the model. In other words, let $P = R^d$ be the parameter space of the model, $M$ the space of all possible morphology for the vascular networks. Also, define the equivalence relation $\sim$ defined on the parameter space $P$ to be
\[ \text{for } p_1, p_2 \in P \wh p_1 \sim p_2 \qquad iff \qquad \mathcal{A}(p_1) \equiv \mathcal{A}(p_2), \]
where $\mathcal{A}: P \to M$ a mapping from the parameter space to the morphology space. The $\equiv$ is yet another equivalence relation defined on the morphology space $M$ where for $m_1, m_2 \in M$ we write $m_1 \equiv m_2$ if and only if $m_1$ and $m_2$ has the same topological characterization. These topological characterizations are computed using the topological data analysis techniques.


\subsection{Method}
Chaplain-Anderson model of angiogenesis used in this paper keeps track of the spatio-temporal evolution of three variables: endothelial tip cells, tumor angiogenesis factor, and fibronectin.

Topological data analysis: Two filtration methods were used: sweeping plane method, and flooding filtration. The filtration is performed on the binary images generated with the Chaplain-Anderson model.


\subsection{Useful facts}
\begin{itemize}
	\item The growth factors the cancer cells release when under low nutritient and oxygen: vascular endothelial growth factor (VEGF), platelet derived growth factor (PDGF), and basic fibroblast growth factor (bFGF).
\end{itemize}

\subsection{Points that are not clear yet}
\begin{enumerate}[(a)]
	\item In the introduction, the authors claim that ``The morphology of a vascular network can reveal the presence of an underlying disease, or predict the response of a patient to treatment'', without any citation of explanation. I think this needs more discussion.
\end{enumerate}

\subsection{Useful papers cited}
\begin{itemize}
	\item Papers related to biology of the tumor induced angiogenesis \cite{Gupta2003,Folkman1971}.
	\item More modern descriptions of the angiogenesis \cite{Lugano2020,Saman2020}
	\item The role of the mechanical stress on the angiogenesis \cite{Li2005,Li2002,Vavourakis2017}
	\item Some old and classic models for the angiogenesis \cite{Anderson1998,Balding1985,Byrne1995,Stokes1991}.
	\item More detailed theoretical models for angiogenesis \cite{Byrne2010,Hadjicharalambous2021,Metzcar2019,Scianna2013}.
	\item Alternative models of angiogenesis \cite{Vilanova2017,Stepanova2021,Perfahl2017,Grogan2017,Vavourakis2017,Cai2017,Sefidgar2015}
	\item Statistical and single scale methods to quantify the vascular networks \cite{Perfahl2017,Folarin2010,Kannan2018,Konerding1999,Konerding2001}
	\item Biological angiogenesis experiments \cite{Bauer2007}
	\item Topological data analysis \cite{Carlsson2009,Zomorodian2005}
	\item Applied Algebraic Topology \cite{2020,2023}
	\item Application of topological data analysis \cite{Nicolau2011,Nielson2015,Qaiser2016,Xia2014}
	\item Topological data analysis and agent based models \cite{Topaz2015,McGuirl2020}
\end{itemize}




\subsection{Results}




	
	
	

% Reset section command back to original after the chapter
\let\section\oldsection
