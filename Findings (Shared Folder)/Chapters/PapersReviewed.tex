\chapter{Papers Reviewed}


% Redefine section command for this chapter
\let\oldsection\section
\renewcommand{\section}{\newreviewsection}

% Your custom sections here

In this section I will keep the notes of the papers I have reviewed, or reproduced their results. 





\section{Topological data analysis distinguishes parameter regimes in the Anderson-Chaplain model of angiogenesis}{Nardini, Byrne}{2021-PLOS CB}


\subsection{Introduction}
This paper studies the Anderson Chaplain \cite{Anderson1998} model of angiogenesis and partitions the parameter spaces based on the morphology of the vascular structure generated by the model. In other words, let $P = R^d$ be the parameter space of the model, $M$ the space of all possible morphology for the vascular networks. Also, define the equivalence relation $\sim$ defined on the parameter space $P$ to be
\[ \text{for } p_1, p_2 \in P \wh p_1 \sim p_2 \qquad iff \qquad \mathcal{A}(p_1) \equiv \mathcal{A}(p_2), \]
where $\mathcal{A}: P \to M$ a mapping from the parameter space to the morphology space. The $\equiv$ is yet another equivalence relation defined on the morphology space $M$ where for $m_1, m_2 \in M$ we write $m_1 \equiv m_2$ if and only if $m_1$ and $m_2$ has the same topological characterization. These topological characterizations are computed using the topological data analysis techniques.


\subsection{Method}
Chaplain-Anderson model of angiogenesis used in this paper keeps track of the spatio-temporal evolution of three variables: endothelial tip cells, tumor angiogenesis factor, and fibronectin.

Topological data analysis: Two filtration methods were used: sweeping plane method, and flooding filtration. The filtration is performed on the binary images generated with the Chaplain-Anderson model.


\subsection{Useful facts}
\begin{itemize}
	\item The growth factors the cancer cells release when under low nutritient and oxygen: vascular endothelial growth factor (VEGF), platelet derived growth factor (PDGF), and basic fibroblast growth factor (bFGF).
\end{itemize}

\subsection{Points that are not clear yet}
\begin{enumerate}[(a)]
	\item In the introduction, the authors claim that ``The morphology of a vascular network can reveal the presence of an underlying disease, or predict the response of a patient to treatment'', without any citation of explanation. I think this needs more discussion.
\end{enumerate}

\subsection{Useful papers cited}
\begin{itemize}
	\item Papers related to biology of the tumor induced angiogenesis \cite{Gupta2003,Folkman1971}.
	\item More modern descriptions of the angiogenesis \cite{Lugano2020,Saman2020}
	\item The role of the mechanical stress on the angiogenesis \cite{Li2005,Li2002,Vavourakis2017}
	\item Some old and classic models for the angiogenesis \cite{Anderson1998,Balding1985,Byrne1995,Stokes1991}.
	\item More detailed theoretical models for angiogenesis \cite{Byrne2010,Hadjicharalambous2021,Metzcar2019,Scianna2013}.
	\item Alternative models of angiogenesis \cite{Vilanova2017,Stepanova2021,Perfahl2017,Grogan2017,Vavourakis2017,Cai2017,Sefidgar2015}
	\item Statistical and single scale methods to quantify the vascular networks \cite{Perfahl2017,Folarin2010,Kannan2018,Konerding1999,Konerding2001}
	\item Biological angiogenesis experiments \cite{Bauer2007}
%	\item Topological data analysis \cite{Carlsson2009,Zomorodian2005}
%	\item Applied Algebraic Topology \cite{2020,2023}
%	\item Application of topological data analysis \cite{Nicolau2011,Nielson2015,Qaiser2016,Xia2014}
%	\item Topological data analysis and agent based models \cite{Topaz2015,McGuirl2020}
\end{itemize}


\section{Quantitative Angiogenesis Assays in vivo – A Review}{J. Hasan}{2004, Angiogenesis}

This paper discusses various angiogenesis assays, highlighting the corneal micropocket and the CAM assay as established methods. It emphasizes the importance of selecting complimentary assays to best replicate tumor angiogenesis for studying the effects of pro- or anti-angiogenic compounds. The development of non-invasive techniques for quantifying angiogenesis is highlighted as a significant advancement for the field \cite{Hasan2004} 


\section{Characterization of lymphocyte-dependent angiogenesis using a SCID mouse: human skin model of psoriasis}{B. Nickoloff}{2000, The journal of investigative dermatology}

This review updates the understanding of angiogenesis in psoriasis, integrating the characterization of endothelial cells in plaques and discussing a novel animal model for triggering neovascularization and plaque formation, providing insights into the angiogenic process in skin disorders \cite{Nickoloff2000}


\section{Integration of experimental and computational approaches to sprouting angiogenesis}{S. Peirce}{2012, Current Opinion in Hematology}

This paper summarizes the integration of experimental tools and computational modeling in studying sprouting angiogenesis, showcasing how such interdisciplinary approaches can lead to new understandings and therapeutic targets by accounting for molecular data and cell-level behaviors \cite{Peirce2012}

\section{Mathematical models of developmental vascular remodelling: A review}{Jessica R. Crawshaw}{2023, PLOS Computational Biology}

Focusing on the less-explored area of developmental remodeling of vascular networks, this review discusses mathematical models that have contributed to understanding the transformation of primitive vessel networks into functional ones, highlighting the multiscale nature of this problem \cite{Crawshaw2023}

\section{Assessment Methods of Angiogenesis and Present Approaches for Its Quantification}{G. J. Khan}{2014, Cancer Research}

This paper provides an overview of angiogenesis assessment methods, including in vitro, in vivo, and in ovo models, focusing on the calculation modes and considerations necessary for concluding the angiogenic or antiangiogenic properties of agents\cite{Khan2014}

\section{Endogenous inhibitors of angiogenesis.}{P. Nyberg, et. al.}{2005, Cancer research}

Highlighting the balance between proangiogenic and antiangiogenic factors, this review explores the role of endogenous inhibitors in the body, providing insight into the potential of leveraging these natural inhibitors for therapeutic purposes in cancer treatment \cite{Nyberg2005}

\section{Mathematical Models of Avascular Tumor Growth}{T. Roose, et. al.}{2007, SIAM Rev}

Offering a comprehensive list and discussion of models for avascular tumor growth, this review emphasizes the importance of mathematical modeling in understanding tumor development and outlines potential future directions for research in this area \cite{Roose2007}

\section{Mathematical Modelling of Angiogenesis}{Chaplain}{2000, Journal of Neuro-Oncology}
discusses a variety of mathematical models used to describe capillary network formation, focusing on a model that generates 2D and 3D vascular structures. This model incorporates the migratory response of endothelial cells to tumor angiogenic factors, cell proliferation, and interactions with extracellular matrix macromolecules, among other factors \cite{Chaplain2000}

\section{Computational and Mathematical Modeling of Angiogenesis}{S. Peirce}{2008, Microcirculation}

reviews mathematical and computational models developed over two decades to study various aspects of angiogenesis. This work emphasizes the insights gained from these models in normal physiological growth, tumorigenesis, wound healing, and therapeutic strategy design \cite{Peirce2008}


\section{Mathematical modelling of angiogenesis using continuous cell-based models}{F. D. Bookholt, et. al}{2016, Biomechanics and Modeling in Mechanobiology}

by Bookholt et al. (2016) introduces a 3D in vitro model simulating early stages of angiogenesis. The model addresses endothelial cell migration due to chemotaxis and durotaxis and includes various proteins impacting angiogenesis \cite{Bookholt2016}

\section{Mathematical modelling of dynamic adaptive tumour-induced angiogenesis: clinical implications and therapeutic targeting strategies}{S. McDougall, et. al.}{2006, Journal of theoretical biology}
presents a model that couples vessel growth with blood flow, offering insights into the adaptive and dynamic nature of tumor-induced angiogenesis and identifying new therapeutic targets for tumor management \cite{McDougall2006}

\section{Angiogenesis—Understanding the Mathematical Challenge}{Pamela F Jones}{2006, Angiogenesis}

Explains the mathematical modelling strategy in biological terms, aiming to bridge the gap between mathematics and life sciences. The paper discusses the assumptions and simplifications foundational to modeling and their implications for understanding angiogenesis \cite{Jones2006}

\section{On the mathematical modeling of wound healing angiogenesis in skin as a reaction-transport process}{Samik Ghosh}{2015, Frontiers in Physiology}

provides a comprehensive review of mathematical models of angiogenesis in skin wound healing. It introduces the continuum reaction-transport framework as a useful tool for exploring unresolved questions in angiogenesis research \cite{Ghosh2015} 

\section{Multiscale Agent-based Model of Tumor Angiogenesis}{Megan M, et. al.}{2013}

Olsen and Siegelmann (2013) developed a three-dimensional multiscale ABM focusing on breast cancer. The model encompasses cellular (genetic control), tissue (cells, blood vessels, angiogenesis), and molecular (VEGF, diffusion) levels. A novel discrete approach to model angiogenesis is proposed to decrease computational cost, offering potential new directions for modeling in cancer research \cite{Olsen2013}

\section{Simulating cancer growth with multiscale agent-based modeling.}{Zhihui Wang, et. al.}{2015, Seminars in cancer biology}
discuss the utility of ABMs in simulating diverse cancer behaviors, including tumor morphology, adaptation to the microenvironment, angiogenesis, and response to therapies. The review highlights the capability of ABMs to simulate the complex interplay between tumor cells and their microenvironment, paving the way for new therapeutic insights \cite{Wang2015}

\section{Agent-based model of angiogenesis simulates capillary sprout initiation in multicellular networks.}{Joseph Walpole, et. al.}{2015, Integrative biology}

present an ABM that incorporates both stochastic and deterministic rules to simulate the initiation of sprouting angiogenesis. The model accurately simulates sprout initiation frequency and location, offering a deeper understanding of the balance between stochasticity and determinism in biological processes \cite{Walpole2015}


\section{Agent-Based Modeling of Vascularization in Gradient Tissue Engineering Constructs}{E. S. Bayarak, et. al.}{2015, IFAC-PapersOnLine}

develop an ABM to simulate vascular growth in engineered biomaterials, investigating the influence of growth factor release rate on angiogenesis. The model's results, validated against experimental studies, suggest microsphere properties that promote angiogenesis, offering insights into tissue engineering applications \cite{Bayrak2015}

\section{A cell-based model exhibiting branching and anastomosis during tumor-induced angiogenesis.}{A. Bauer, et. al.}{2007, Biophysical journal}
describe a cell-based ABM that integrates endothelial cell migration, growth, division, and the evolving structure of the stroma at the cellular Potts model level. The model successfully reproduces various morphologies of capillary sprouts observed in vivo, demonstrating the emergence of branching and anastomosis without prescribed rules \cite{Bauer2007a}


\section{Coupled mathematical model of tumorigenesis and angiogenesis in vascular tumours}{M. Cooper, et. al.}{2010, Cell Proliferation}

developed a model that combines the processes of avascular tumor growth and the development of capillary networks through tumor-induced angiogenesis. This comprehensive model offers insights into the growth and development mechanisms of vascular tumors \cite{Cooper2010}

\section{Tree topology analysis of the arterial system model}{V. Kopylov}{2018, Journal of Physics}

presented an algorithm for constructing an arterial system model with physiologically significant geometric properties. Their analysis of the bifurcation exponent's effect on the arterial network's topology provides valuable insights into the optimal network topology for efficient vascular function  \cite{Kopylova2018}

\section{Mathematical Model of Blood Flow in an Anatomically Detailed Arterial Network of the Arm}{Sansuke M, el. al.}{2013, Mathematical Modelling and Numerical Analysis}

Watanabe, Blanco, and Feijóo (2013) developed a detailed model for hemodynamics simulations in the arm's arterial network. Their model includes a comprehensive arterial topology and offers a systematic estimation of involved parameters, allowing for accurate simulations of blood flow and pressure \cite{Watanabe2013}


\section{An integrated approach to quantitative modelling in angiogenesis research}{A. J. Connor, et. al.}{2015, Journal of The Royal Society Interface}

discuss a multidisciplinary approach combining experiments, image processing, analysis, and mathematical modeling focused on angiogenesis in the cornea micropocket assay. This approach aims to provide mechanistic insights into the action of angiogenic factors through quantitative data extraction and model parametrization \cite{Connor2015}


\section{Integration of experimental and computational approaches to sprouting angiogenesis}{S. Peirce, et. al.}{2012, Current Opinion in Hematology}

Peirce et al. (2012) summarize recent advancements in computational modeling of angiogenesis, driven by detailed molecular data and experimental tools. These models help predict hypothetical experiment outcomes and generate new hypotheses for understanding angiogenesis at a system-wide level \cite{Peirce2012}

\section{A Computational Tool for Quantitative Analysis of Vascular Networks}{E. Zudaire, et. al.}{2011, PLoS ONE}
developed AngioTool, a user-friendly software for the quantification of vascular networks in microscopic images. AngioTool computes several morphological and spatial parameters and is open source, available for free download, facilitating standardized analysis in angiogenesis research \cite{Zudaire2011}

\section{Consensus guidelines for the use and interpretation of angiogenesis assays}{Many authors}{2018, Angiogenesis}
published the first edition of consensus guidelines for the use and interpretation of angiogenesis assays. This collaborative work aims to serve as a reference for current and future angiogenesis research, promoting standardized methodologies across the field \cite{NowakSliwinska2018a}

\section{Zebrafish as an Emerging Model Organism to Study Angiogenesis in Development and Regeneration}{Myra N Chávez, et. al.}{2016, Front Physiol}
ocus on the zebrafish (Danio rerio) as an emerging model organism for studying angiogenesis in development and regeneration, highlighting its potential for understanding vascularization in artificial tissues and organs, as well as for drug discovery \cite{Chavez2016}




	
	
	

% Reset section command back to original after the chapter
\let\section\oldsection
